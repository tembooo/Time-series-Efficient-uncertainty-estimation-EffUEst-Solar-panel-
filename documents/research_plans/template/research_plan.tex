%% $Id$

%\documentclass[pdftex,12pt,a4]{article}
\documentclass[11pt,a4,notitlepage]{article}

% Packages
\usepackage[utf8]{inputenc}
\usepackage[T1]{fontenc}

%\usepackage[nolist]{acronym}
\usepackage{algorithmic}
\usepackage{amsmath}
%\usepackage{amssymb}
%\usepackage{cite}
\usepackage[usenames]{color}
%\usepackage{csvtools} % obsolete package
\usepackage{datatool} % replaces csvtools
\usepackage{draftcopy}
\usepackage{epsfig}
\usepackage{eurosym}
\usepackage{geometry}
\usepackage[acronym,shortcuts]{glossaries}
%\usepackage[pdftex]{graphicx}
\usepackage{graphicx}
%\usepackage{html}
\usepackage[pdfborder={0 0 1}]{hyperref}
\usepackage{mfirstuc}
%\usepackage{odsfile}
\usepackage{pslatex}
\usepackage{relsize}
\usepackage{SIunits}
\usepackage[tight]{subfigure}
%\usepackage{subfigure}
\usepackage{theorem}
\usepackage{times}
%\usepackage[obeyDraft]{todonotes}
\usepackage{url}
\usepackage{wrapfig}

\DeclareGraphicsExtensions{.eps,.pdf,.png,.jpg}

% natbib must be loaded before babel to prevent tilde showing up in
% et~al.
%\usepackage[square]{natbib}
\usepackage[english]{babel}

% Special to a research plan
\usepackage{../resources/research_plan}
%\usepackage{research_plan}

% Settings
\draftcopyName{Confidential}{130}
\geometry{verbose,a4paper,nohead,tmargin=20mm,bmargin=20mm,lmargin=20mm,rmargin=20mm}
\setcounter{secnumdepth}{5}
\setcounter{tocdepth}{3}
\setlength\parskip{0.5\medskipamount}
\setlength\parindent{0pt}
\subfigcapskip=0pt
\subfigbottomskip=0pt

%\graphicspath{{resources/}}

%\begin{acronym}
\newacronym{CVPRL}{CVPRL}{Computer Vision and Pattern Recognition Laboratory}
\newacronym{CVS}{CVS}{concurrent version system}
\newacronym{EBM}{EBM}{evidence-based medicine}
\newacronym{GAN}{GAN}{generative adversarial network}
\newacronym{GT}{GT}{ground truth}
\newacronym{ICP}{ICP}{iterative closest point}
\newacronym{IQA}{IQA}{image quality assessment}
\newacronym{IQM}{IQM}{image quality metric}
\newacronym{IQMs}{IQMs}{image quality metrics}
\newacronym{LUT}{LUT}{Lappeenranta University of Technology}
\newacronym{NR}{NR}{no-reference}
\newacronym{RGB}{RGB}{red-green-blue}
\newacronym{RR}{RR}{reduced-reference}
\newacronym{SVN}{SVN}{Subversion}
%\end{acronym}

\newcommand{\researcher}{N. N.}

% RESEARCH PLAN OUTLINE (www.aka.fi)
%
% The length of a research plan must not exceed 12 pages (Times New Roman
% 12 pt or corresponding). The research plan shall include the following
% information, intended for an international expert for purposes of
% evaluating the application:
%%%%%%%%%%%%%%%%%%%%%%%%%%%%%%%%%%%%%%%%%%%%%%%%%%%%%%%%%%%%%%%%%%%%%%%%%%%%%%%%
\begin{document}
\pagestyle{plain}
\textbf{Research plan (Confidential)} \hfill Lappeenranta
\number\year-\number\month-\number\day

%%%%%%%%%%%%%%%%%%%%%%%%%%%%%%%%%%%%%%%%%%%%%%%%%%%%%%%%%%%%%%%%%%%%%%%%%%%%%%%%
\begin{center} \begin{tabular}{rp{0.666\textwidth}}
%
Topic: & \textbf{PROJECT TITLE (ACRONYM)} \\
%
Related to: & RESEARCH FOCUS AREA (ACRONYM) \\
%
Resources: & \researcher , DAY.MONTH.YEAR--DAY.MONTH.YEAR \\
%
Results: & \resulta{Background information}, \\
%
& \resulta{METHOD DEVELOPMENT}, \\
%
& \resulta{EVALUATION}, and \\
%
& \resulta{Documentation}. \\
%
\end{tabular} \end{center} \vspace{2\medskipamount}

%%%%%%%%%%%%%%%%%%%%%%%%%%%%%%%%%%%%%%%%%%%%%%%%%%%%%%%%%%%%%%%%%%%%%%%%%%%%%%%%
\begin{center} \small \begin{tabular}{llp{0.5\textwidth}l}
%
\textbf{Date} & \textbf{Author} & \textbf{Comments} & \textbf{Status}\\
%
\hline
%
YEAR-MONTH-DAY & LTL & Initial version & Draft \\
%
\end{tabular} \end{center}

\vspace{2\medskipamount}

%%%%%%%%%%%%%%%%%%%%%%%%%%%%%%%%%%%%%%%%%%%%%%%%%%%%%%%%%%%%%%%%%%%%%%%%%%%%%%%%
\section{Background}

% Evidence-based medicine and medical images
\Ac{EBM} is the current practise in many subfields of medical science. In this
approach, the medical diagnosis and planning of treatment is based on scientific
knowledge and objective examination of the patient through biomedical
measurements. One example of the knowledge used in the process is images because
of the versatile possibilities to image the condition of the patient or her
organs. From this viewpoint, the medical doctors base their decisions nowadays
on a more complete and timely view to the situation.

% Eye diseases and current imaging
Eye diseases have become one of the rapidly increasing health threats worldwide.
For example, diabetes causes abnormalities in the retina (diabetic retinopathy),
kidneys (diabetic nefropathy), and nervous system (diabetic neuropathy). The
diabetic retinopathy and other eye-related diseases are diagnosed from eye
fundus images (Fig.\,\ref{fig:images}) by medical experts who look for special
lesions in the images. The attention of medical expert in fundus examination
restricts the possibility to perform broad screenings. For screening and
monitoring a progressive disease, automatic image processing methods are a
well-motivated possibility to help a single expert's work, or enable a wider
screening program.
%
\begin{figure}[!ht] \centering
%
\subfigure[Missing
image.]{\includegraphics[width=0.4\textwidth]{resources/missing}
\label{fig:image1}}
%
\subfigure[Missing
image.]{\includegraphics[width=0.4\textwidth]{resources/missing}
\label{fig:image2}}
%
\caption{Example images.} \label{fig:images}
%
\end{figure}

% ...

%%%%%%%%%%%%%%%%%%%%%%%%%%%%%%%%%%%%%%%%%%%%%%%%%%%%%%%%%%%%%%%%%%%%%%%%%%%%%%%%
\section{Objectives}

The objectives of this research are as follows:
%
\begin{enumerate}
%
\item Carry out a literature search and write a review about the findings.
%
\item Study the available methods for ...
%
\item Implement a selected method for ...
%
\item Evaluate the method performance with the existing data, and analyse the
results.
%
\item Document the work in the form of a ...
%
\end{enumerate}

%%%%%%%%%%%%%%%%%%%%%%%%%%%%%%%%%%%%%%%%
\section{Research methods and material}

%%%%%%%%%%%%%%%%%%%%%%%%%%%%%%%%%%%%%%%%
\subsection{Research methods}

In this reasearch, the focus is on ...  The required research methods and
evalution procedures arise from the preceding research of \ac{CVPRL} and, when
appropriate, the standard ones within the research field(s). In addition,
capability to do independent work (under supervision) and good skills in the
English language are needed.

%%%%%%%%%%%%%%%%%%%%%%%%%%%%%%%%%%%%%%%%%%%%%%%%%%%%%%%%%%%%%%%%%%%%%%%%%%%%%%%%
\subsection{Research steps}

The main steps of the work are given below.

\begin{enumerate}
%%%%%%%%%%%%%%%%%%%%%%%%%%%%%%%%%%%%%%%%
\item[] \textbf{Background}
%
\item If not familiar, study the version control system used in the project.
Start using the repository/directory specified by the supervisor(s) and commit
your changes daily. For the data used in the research, results produced in the
experiments, and literature produced by others, use the file server (not the
version control repository) as instructed.
%
\item If not familiar, study documentation with LaTex and BibTex. Take the
existing template and update the manuscript as the research progresses.
%
%\item If not familiar, study the capabilities of relevant Matlab toolboxes.
%
\item Study the preceding relevant work and the available
data accessible to \ac{CVPRL}.
%
\item Review the relevant in-context literature and acquire at least an initial
understanding of ...
%
\item Write a review of the found methods to the manuscript.
%
\result{Background information}

%%%%%%%%%%%%%%%%%%%%%%%%%%%%%%%%%%%%%%%%
\item[] \textbf{METHOD DEVELOPMENT}
%
\item Review the literature on ...
%
\item Study ...
%
\item Initially experiment with ...
%
\item After selecting the approaches to use with the supervisor(s), implement
the methods.
%
\item Plan and carry out validation experiments on the available data.
%
\item Document the methods and results in the thesis manuscript.
%
\result{Alignment of image and model spectra}

%%%%%%%%%%%%%%%%%%%%%%%%%%%%%%%%%%%%%%%%
\item[] \textbf{EVALUATION}
%
\item Study ...
%
\item Design the experiments based on ...
%
\item Carry out the performance evaluation experiments with the available data,
and analyse the results.
%
\item Document the methods and results in the thesis manuscript.
%
\result{Inversion of spectral images}

%%%%%%%%%%%%%%%%%%%%%%%%%%%%%%%%%%%%%%%%
\item[] \textbf{Documentation}
%
\item Complement the thesis manuscript as needed.
%
\item Get comments from the supervisor(s).
%
\item Finalise the manuscript according to the comments.
%
%\item If the research results allow, prepare a manuscript to a scientific
%conference (or journal) based on the Master's thesis.
%
\result{Documentation}

\end{enumerate}

%%%%%%%%%%%%%%%%%%%%%%%%%%%%%%%%%%%%%%%%%%%%%%%%%%%%%%%%%%%%%%%%%%%%%%%%%%%%%%%%
\section{Implementation}

%%%%%%%%%%%%%%%%%%%%%%%%%%%%%%%%%%%%%%%%
\subsection{Timetable}

The tentative schedule for the work is shown in Table\,\ref{tab:timetable}.  The
relevant notes are as follows:
%
\begin{itemize}
%
\item The order of the topics in the schedule should not be understood as fixed.
Some topics can be studied in parallel.
%
\item If instructed by the project manager, all the working hours must be put
into the work time management system.
%
\item There will be a full evaluation of the results after the research work.
Possibilities for further publication of the results will be considered.
%
\end{itemize}

\renewcommand{\dtldisplayafterhead}{\hline}
%
\DTLloadrawdb{timetable}{research_resources.csv}
%
\begin{table}[h] \centering
%
\caption{Timetable for the research.} \label{tab:timetable}
%
\DTLdisplaydb{timetable}
%
\end{table}

%%%%%%%%%%%%%%%%%%%%%%%%%%%%%%%%%%%%%%%%%%%%%%%%%%%%%%%%%%%%%%%%%%%%%%%%%%%%%%%%
\section{Researchers and research environment}

The following researchers will take part to the research:

\begin{center} \begin{tabular}{lll}
%
Position & Name & Notes \\
%
\hline
%
Research assistant & \researcher & \\
%
%1st supervisor     & Prof. Lasse Lensu & \\
%
%2nd supervisor     & MSc Lauri Laaksonen & \\
%
\hline
%
%Thesis examiner & Prof. Lasse Lensu & \\
%
%Thesis 2. examiner & N. N. & \\
%
\end{tabular} \end{center}

%%%%%%%%%%%%%%%%%%%%%%%%%%%%%%%%%%%%%%%%%%%%%%%%%%%%%%%%%%%%%%%%%%%%%%%%%%%%%%%%
\section{Mobility}

This research requires collaboration with experts working on different fields
of science. Research visits in this regard are performed as needed.

%%%%%%%%%%%%%%%%%%%%%%%%%%%%%%%%%%%%%%%%%%%%%%%%%%%%%%%%%%%%%%%%%%%%%%%%%%%%%%%%
\section{Researcher training}

\researcher\ will get experience on research work within the field of image
processing. The project strengthens the medical image processing research in
\ac{CVPRL} and supports the specialisation of the supervisor(s) in the topic.

%%%%%%%%%%%%%%%%%%%%%%%%%%%%%%%%%%%%%%%%%%%%%%%%%%%%%%%%%%%%%%%%%%%%%%%%%%%%%%%%
\section{Expected results and risks}

The expected results are as follows:
%
\begin{itemize}
%
\item Literature reviews on selected methods for ...
%
\item A method for ...
%
\item Experimental results in the form of ...
%
\item The documentation and notes for future work.
%
\end{itemize}

The potential risks include the following:
%
\begin{itemize}
%
\item Due to the challenging topic, the timetable planned for the research is
unrealistic. It means that all the objectives may not be fully achieved.
%
\item The literature review brings up research which affects the novelty of this
research. This would affect the plans to publish the scientific results.
%
\end{itemize}

%%%%%%%%%%%%%%%%%%%%%%%%%%%%%%%%%%%%%%%%%%%%%%%%%%%%%%%%%%%%%%%%%%%%%%%%%%%%%%%%
\bibliographystyle{acm}
\bibliography{research_plan}

%%%%%%%%%%%%%%%%%%%%%%%%%%%%%%%%%%%%%%%%%%%%%%%%%%%%%%%%%%%%%%%%%%%%%%%%%%%%%%%%
\appendix

\section{Notes about literature surveys}\label{sec:literature_survey}

A few notes for starting a literature survey:
%
\begin{enumerate}
%
\item Select initially a focused set of keywords for literature searches. If you
end up with too few items in the searches, edit the keywords to become more
general. Update the set of keywords as needed.
%
\item To make literature searches, make use of the following:
%
\begin{itemize}
%
\item \href{http://scholar.google.com}{Google Scholar}.
%
\item The publisher databases accessible through LUT library such as
\href{https://ieeexplore.ieee.org/}{IEEE Xplore},
\href{http://www.scopus.com/home.url}{Scopus} and
\href{https://link.springer.com/}{Springer}.
%
\end{itemize}
%
\item Focus on articles that have a high number of citations and/or have been
published in top journals or top conferences.
%
\item Read the abstracts and conclusions to be able to say whether each article
is relevant for your research.
%
\item When you find an interesting paper:
%
\begin{itemize}
%
\item If the paper is an author's manuscript, for example, in arxiv,
ResearchGate or something similar, check whether the article has been properly
published and use that version.
%
\item Go through the list of references of that paper to find the earlier works
on the topic.
%
\item Use Google Scholar or another database to get a list of papers that cite
the paper and review them.
%
\item For significant research groups related to your research, go through the
publication lists of the authors.
%
\item Store the bibliography information and electronic article copies to a
bibliography management system such as \href{http://www.zotero.org/}{Zotero}.
%
\end{itemize}
%
\end{enumerate}

\end{document}
